\documentclass{easychair}

% \usepackage{doc}
\usepackage{setspace}
\usepackage{verbatim}
\usepackage{amssymb}

%----Making things more compact
\newcommand{\smalltt}[1]{\small \texttt{#1}}
\newenvironment{packed_itemize}{
\vspace*{-0.3em}
\begin{itemize}
\setlength{\partopsep}{0pt}
\setlength{\itemsep}{1pt}
\setlength{\parskip}{0pt}
\setlength{\parsep}{0pt}
}{\end{itemize}}
\newenvironment{packed_enumerate}{
\vspace*{-0.3em}
\begin{enumerate}
\setlength{\partopsep}{0pt}
\setlength{\itemsep}{1pt}
\setlength{\parskip}{0pt}
\setlength{\parsep}{0pt}
}{\end{enumerate}}
% \renewcommand{\textfraction}{0.07}
% \renewcommand{\topfraction}{0.9}
% \renewcommand{\bottomfraction}{0.9}
% \renewcommand{\floatpagefraction}{0.66}
% \setlength{\floatsep}{2.0pt plus 2.0pt minus 2.0pt}
% \setlength{\textfloatsep}{5.0pt plus 2.0pt minus 0.0pt}

\title{The New TPTP Format for \\ Tarskian and Kripke Interpretations}

\author{
  Geoff Sutcliffe\inst{1}
\and
  Alexander Steen\inst{2}
\and
  Pascal Fontaine\inst{3}
}

\institute{
  University of Miami,
  Miami, USA\\
  \email{geoff@cs.miami.edu,jam771@miami.edu}
\and
  University of Greifswald,
  Greifswald, Germany\\
  \email{alexander.steen@uni-greifswald.de}
\and
  University of Li{\`e}ge,
  Li{\`e}ge, Belgium\\
  \email{Pascal.Fontaine@uliege.be}
}

\authorrunning{Sutcliffe, Steen, Fontaine}
\titlerunning{TPTP World Interpretations}

\begin{document}
\maketitle

%--------------------------------------------------------------------------------------------------
\begin{abstract}
This paper describes a new format for representing Tarskian and Kripke interpretations.
%formulae in untyped and typed first-order logic, using the TPTP TF0 language.
\end{abstract}
%--------------------------------------------------------------------------------------------------
\section{Introduction}
\label{Introduction}

Historically, Automated Theorem Proving (ATP) has, as the name suggests, focused largely on the
task of proving theorems from axioms -- the derivation of conclusions that follow inevitably 
from known facts \cite{RV01-HAR}.
The axioms and the conjecture to be proved (and hence become a theorem) are written in an 
appropriately expressive logic, and the proofs are often similarly written in logic \cite{SS+06}.
In the last two decades the converse task of disproving conjectures, i.e., proving that a 
conjecture is not a theorem of the axioms, has become increasingly important.
This process depends on finding a {\em countermodel} for the conjecture, i.e., an 
{\em interpretation} (a structure that maps formulae to truth values) that is a {\em model}
of the axioms (maps the axioms to $true$) but not a model for the conjecture (maps the conjecture
to $false$).
A salient application area that uses this form of ATP is verification \cite{DKW08}, where a 
countermodel is used to pinpoint the reason why a proof obligation fails, and correspondingly 
points to the location of the fault in the system being verified.
Other applications of model finding include checking the consistency of an axiomatization 
\cite{SS+17}, and finding a solution to a problem that is coded as a model finding problem 
\cite{Win82}.

The TPTP World \cite{Sut17} (\href{https://www.tptp.org}{\tt www.tptp.org}) is a well established 
infrastructure that supports research, development, and deployment of 
% Automated Theorem Proving 
ATP systems.
Various parts of the TPTP World have been deployed in a range of applications, in both academia 
and industry.
The TPTP World includes the TPTP problem library \cite{Sut09}, 
the TSTP solution library \cite{Sut10}, 
tools and services for processing ATP problems and solutions \cite{Sut10}, 
it supports the CADE ATP System Competition (CASC) \cite{Sut16}.
The TPTP language \cite{Sut23-IGPL} is one of the keys to the success of the TPTP World.
Originally the TPTP World supported only first-order clause normal form (CNF)
\cite{SS98-JAR}.
Over the years full first-order form (FOF)
\cite{Sut09}, 
typed first-order form (TFF)
\cite{SS+12,BP13-TFF1}, 
typed extended first-order form (TXF)
\cite{SK18}, 
typed higher-order form (THF)
\cite{SB10,KSR16}, 
and non-classical forms (NTF).
Most relevant to this work, the TPTP languages are used for writing ATP problems, 
derivations, and interpretations \cite{SS+06,Sut08-KEAPPA}.
Examples of problems are in Appendices~\ref{FOF_Finite.p}, \ref{TFF_Finite.p},
\ref{TFF_Infinite.p}, \ref{THF_Finite.p}, \ref{NTF_Finite-Finite-Global.p}, 
and~\ref{NTF_Finite-Finite-Local.p}.

A TPTP format for interpretations with finite domains was previously been defined \cite{SS+06},
and has served the ATP community adequately for almost 20 years. 
The old format is output by several ATP systems, e.g., Paradox \cite{CS03}, FMDarwin \cite{BF+06}, 
Vampire \cite{KV13}.
Recently the need for a format for interpretations with infinite domains, and for a format for 
Kripke interpretations \cite{Kri63}, has led to the development of a new TPTP format for 
interpretations.
This work describes the new format.
The underlying principle is unchanged: interpretations are represented in formulae.

\paragraph{Related Work:}
There are other concrete representations of interpretations in use:
The SMT-LIB standard \cite{BFT17} defines a format for model output, and commands to inspect 
models.  
SAT solvers generally output models as specified by the SAT competitions \cite{JL+12}, in a 
simple format similar to the DIMACS input format \cite{Bab93}.
Some individual model finding systems have defined their own formats for models, e.g., the 
output formats of Nitpick and Z3 \cite{dMB08}.
% +++
% Nikolaj says ...
% Z3 It produces models that define functions by expressions. For example a model of succ is 
% Succ(x) =X+ 1
% Works when domain is integer. Currently z3 does not implement infinite models for uninterpreted sorts. I would probably support infinite sorts by creating injection into an algebraic datatype and then support models that can be expressed over ADT.
% See also https://microsoft.github.io/z3guide/docs/logic/Quantifiers
% +++

\vspace*{1em}
This paper is organized as follows:
Section~\ref{Interpretations} discusses the nature of interpretations, considering what is
needed from interpretations, and the various forms that interpretations can take.
Section~\ref{NewTarskian} defines the new format for Tarskian interpretations, and
Section~\ref{NewKripke} does the same for Kripke interpretations.
% Section~\ref{Verification} describes the semantic approach to model verification.
Section~\ref{Conclusion} concludes and discusses plans for future work.

%--------------------------------------------------------------------------------------------------
\section{About Interpretations}
\label{Interpretations}

%--------------------------------------------------------------------------------------------------
\subsection{What do we Need?}
\label{Need}

The needs of applications that use model finding vary according to their use of the model.
In the simplest case application need only to know that a model exists.
Examples of such applications include checking the consistency of an axiomatization \cite{CI15},
use as a subroutine in more complex reasoning, e.g., for
axiom selection \cite{SP07,Pud07-ESARLT}, and establishing the existence of a bug in a
verification process\footnote{%
Bill McCune claimed that establishing the existence of a bug without having an explicit model
to help pinpoint the bug would be ``frustrating''. And he should have known.}.
A key weakness of model finder systems that claim to have found a model but do not output an 
explicit model is that it is necessary to trust the model finder.

In many applications it is necessary to have an explicit model, in some representation format that
allows for analysis of the model.
Applications that productively use an explicit model include finding inconsistencies in 
axiomatizations \cite{SS+17}, identification of bugs in verification \cite{CE82,QS82},
solution of problems encoded as a model finding problems \cite{Win82}, and evaluating formulae
wrt the model \cite{SS+23-LPAR}.
Manual inspection of explicit models can also be useful, e.g., \cite{EK+10}.
Explicit models can be used for machine learning and improving model finders internally.
A key advantage of having an explicit model output is that the model can be verified, i.e., the
model finder does not need to be trusted,

Given the innate desirability of obtaining explicit models of satisfiable formulae, desirable
properties of interpretation representation can be considered.
\begin{packed_itemize}
\item Verification of models (checking that the formulae evaluate to $true$ wrt the model), 
      should be possible.
\item Evaluation of formulae wrt an interpretation should be tractable.
\item Interpretations should be sufficiently comprehensible for manual inspection.
\end{packed_itemize}

%--------------------------------------------------------------------------------------------------
\subsection{What do we Have?}
\label{Have}

A {\em Tarskian interpretation} \cite{TV56} of formulae in first-order logic consists of a 
non-empty domain of unequal elements for each type used in the formulae (just one domain for 
untyped logic), and mappings for the function and predicate symbols with respect to the 
domains \cite{Hun96,Gal15}.
An overview of some ways of building and representing Tarskian interpretations is provided 
by \cite{CLP04}.
A {\em complete} interpretation can interpret all expressions that can be written in the language 
of the formulae, but a {\em partial} interpretation can interpret only (at least) the 
given formulae, e.g., \cite{BSW23}.
Two types of Tarskian interpretations are clear (and more might exist)~\ldots
\begin{packed_itemize}
\item {\em Finite} Tarskian interpretations have only finite domains.
      The domain and symbol mappings can be be explicitly enumerated.
      Finite models for a set of formulae are typically produced by starting with domains of 
      size one, and incrementing the sizes until a model is found.
      At each iteration the formulae are reduced to either propositional 
      \cite{CS03,McC03-MACE4-TR} or function free \cite{BF+09} logic, both of which are decidable.
      An ATP system can then decide if there is a model.
      There are several ATP systems that produce finite Tarskian models, e.g., Paradox, FMDarwin, 
      and Vampire.
\item {\em Infinite} Tarskian interpretations have one or more infinite domains.
      Infinite domains can be explicitly generated (e.g., terms representing Peano numbers), 
      or implicitly specified (e.g., some set of algebraic numbers, such as the integers).
      % HOW IS THIS DONE?
      There are some ATP systems that produce infinite Tarskian models, e.g., 
      cvc5 \cite{BB+22-cvc5} and Z3.
\end{packed_itemize}
\vspace*{-0.5em}
Formulae can be interpreted wrt (finite) Tarskian interpretations~\ldots
\begin{packed_itemize}
\item Interpret quantifiers using Tarskian semantics.
\item Interpret ground (grounded with domain elements) terms and atoms using the mappings.
\item Interpret boolean formulae using the truth tables for connectives.
\end{packed_itemize}
\vspace*{-0.5em}
This approach is taken internally in Vampire.

\vspace*{0.8em}
A {\em Herbrand interpretation} \cite{Her30} has the Herbrand universe as the domain, the mapping 
for non-boolean symbols (functions) is the ``identity'', and the mapping for boolean symbols 
(predicates) is from the Herbrand base to $\{true,false\}$.
Every set of formulae induces its set of Herbrand models.
Such sets can be the result from an ATP system applying model preserving transformations to its 
input (even the input iteself induces its set of Herbrand models), or be generated by an ATP 
system with the explicit intention of representing Herbrand models:
\begin{packed_itemize}
\item {\em Saturations} \cite{BG+01} are a fixed point for a set of clauses at which further 
      application of a complete inference system does not generate any new clauses.
      This approach is adopted in saturation-based ATP systems such as E \cite{SCV19},
      Prover9 \cite{McC-Prover9-URL}, Vampire, and Zipperposition \cite{VB+21}.
      While the domain of the Herbrand interpretations induced by a saturation is known 
      to be the Herbrand Universe, there is no explicit symbol interpretation that can be 
      used constructively by users.
\item {\em Formulae} that are intended to represent Herbrand interpretations, written 
      {\em Herbrand-formulae} to avoid confusion, can be produced.
      Examples include a disjunction of implicit generalisations (DIGs) \cite{LM87}, and 
      predicate definitions over the term algebra \cite{SK12}.
      iProver \cite{Kor08,SK12} is an ATP system that outputs the latter format.
\end{packed_itemize}

Ways to make Herbrand interpretations:
\begin{packed_itemize}
\item Tableaux. as long as you use a proof confluent calculus, it is normally possible to 
      construct a model from a saturated branch \cite{Hah01}.
\item Hyper-tableaux. \cite{BFN96,Bau98,BFP07}
\item Ordered Semantic Hyper-linking \cite{PZ00} NEED TO READ
\item SGGS \cite{BP16}
\item Proving Infinite Satisfiability \cite{BB13}
\item Model Evolution with Equality \cite{BT03,BFT04,BT05,BPT11}
\item EQMC \cite{CP00,Pel03-EQMC}
\item Saturation \cite{Pel03-JSC}
\item RAMC \cite{CZ92,CP95,CP95-TAB}
\end{packed_itemize}

A {\em Kripke interpretation} \cite{Kri63} adds a layer of {\em worlds} over Tarskian 
interpretations.
There can be a finite or infinite number of worlds.
There is an {\em accessibility relation} between the worlds, which can be subject to the
requirements of the logic being used, e.g., for modal logic {\bf M} the accessibility 
relation must be reflexive.
Within each world there is a Tarskian interpretation, and there can be some interaction
between the worlds' Tarskian interpretations if the terms designate globally MORE TO BE SAID.
Formulae can be interpreted wrt Kripke interpretations (with finite worlds and local terms)~\ldots
\begin{packed_itemize}
\item Interpret quantifiers over worlds using Kripke semantics.
\item Interpret formulae within a world wrt the Tarskian interpretation in the world.
\end{packed_itemize}

The notions of interpretations, models, partial interpretations, finite interpretations,
Herbrand interpretations, etc., are captured in the SZS ontologies \cite{Sut08-KEAPPA}, as
updated at 
\href{https://www.tptp.org/cgi-bin/SeeTPTP?Category=Documents\&File=SZSOntology}{\tt www.tptp.org}
\href{https://www.tptp.org/cgi-bin/SeeTPTP?Category=Documents\&File=SZSOntology}{\tt /cgi-bin/SeeTPTP?Category=Documents\&File=SZSOntology}.
For this work the ontology was extended with a new value {\em Model Preserving} (MPR), defined
as ``Some interpretations are models of Ax, and
  all models of Ax are extended to a model of C, and
  all models of C are an extension of a model of Ax
  (which means that all models of C are models of Ax)''.
This is used for transformations on satisfiable sets of formulae in which the models of the set 
are unchanged, or are changed only by adding new domain elements or mappings, so that the
extended models are still models of the original set of formulae. 
Examples of such transformations are adding logical consequence of the set into the set, and
Skolemization.

%--------------------------------------------------------------------------------------------------
\subsection{Do we Have what we Need?}
\label{HaveNeed}

The new TPTP format represents Tarskian and Herbrand interpretations in 
{\em interpretation-formulae} -- the details are provided in Sections~\ref{NewTarskian} 
and~\ref{NewKripke}.
Interpretation-formulae are written in an extension of the language of the formulae being 
interpreted, i.e., extending the vocabulary of the formulae, and using the same syntax.

Figure~\ref{ModelLandscape} provides an overview of the situation.
The starting point is the set of {\sf Satisfiable formulae in $\Sigma$}, which might have been
formed from {\sf Ax $\cup$ \{{\raisebox{0.4ex}{\texttildelow}}C\}}. 
$\Sigma$ is the language of the formulae.
Going down leads to the {\sf Models} of the satisfiable formulae.
Some of the models are Tarskian models, and some are Herbrand models.
Going left from the {\sf Satisfiable formulae in $\Sigma$} is the pathway taken by ATP systems 
that find a Tarskian model, and output interpretation-formulae representing the model.
If the interpretation-formulae correctly represents a model of the satisfiable formulae, then
the satisfiable formulae can be proved ($\vDash$) from the interpretation-formulae.
The models of the interpretation-formulae are a subset ($\subseteq$) of the models of the
{\sf Satisfiable formulae in $\Sigma$}.
Going right from the {\sf Satisfiable formulae in $\Sigma$} is the pathway taken by ATP systems
that apply model preserving transformations ({\sf MPR}s) on the satisfiable set, possibly 
including conversion to CNF with Skolemization, to produce a saturation that induces a set of 
Herbrand models that are models of the original {\sf Satisfiable formulae in $\Sigma$}.
% (Transformations such as these might also be applied as part of the process of building a 
% Tarskian interpretation formula.)
Going right and down from the {\sf Satisfiable formulae in $\Sigma$} is the pathway taken by ATP 
systems that generate Herbrand-formulae that induce a set of Herbrand models that are models of 
the original {\sf Satisfiable formulae in $\Sigma$}.
The {\sf Satisfiable formulae in $\Sigma$} can be proved from each of the sets that result from 
applying MPRs

\begin{figure}[htbp]
\centering
\includegraphics[width=0.75\textwidth]{ModelLandscape.pdf}
\caption{A Landscape of Classical Logic Model Building}
\label{ModelLandscape}
\end{figure}

To interpret a formula wrt interpretation-formulae~\ldots
% $\varphi$ wrt a given interpretation $I$ ($I\,\vdash\,\Phi$ means that 
% $\Phi$ is $true$ in $I$, i.e., $I$ is a model of formula $\Phi$)~\ldots
\begin{packed_itemize}
\item For Kripke interpretation-formulae, theorem proving can be used.
      RIGHT ALEX - THAT's WHAT AGMMV DOES FOR TFF?
\item For Tarskian interpretation-formulae, theorem proving can be used.
      \begin{packed_itemize}
      \item If a formula can be proved from the interpretation-formulae then it is $true$ in the 
            interpretation represented by the interpretation-formulae.
            The correctness of this is shown for finite Tarskian interpretations for untyped 
            first-order logic in \cite{SS+23-LPAR}, by proving that if a formulae $\Phi$ can be 
            proved from the interpretation-formulae $\varphi$ then the interpretation $I$ 
            represented by $\varphi$ is a model of $\Phi$.
            We believe the theorem lifts naturally to TFF and THF, but is technically complicated 
            due to the introduction of types.
            % and type promotion functions.
            The extension to infinite domains is quite simple after that.
      \item If a formula can be disproved from the interpretation-formulae then it is $false$ in 
            the interpretation represented by the interpretation-formulae.
      \item In practice, a formula might be neither proved nor disproved within reasonable 
            resource limits, so that nothing is known.
      \end{packed_itemize}
\item For Herbrand interpretations induced by saturations or Herbrand-formulae, theorem proving 
      can be used~\ldots
      \begin{packed_itemize}
      \item If a formula can be proved from the saturation or Herbrand-formulae then it is 
            $true$ in all the Herbrand interpretations induced by the saturation or 
            Herbrand-formulae.
      \item If a formula can be disproved from the saturation or Herbrand-formulae then it is 
            $false$ in in all the Herbrand interpretations induced by the saturation or 
            Herbrand-formulae.
      \item In practice, a formula might be neither proved nor disproved within reasonable 
            resource limits, so that nothing is known.
      \end{packed_itemize}
      I believe that this verification technique works, and so do Stephan Schulz and Uwe Waldmann,
      but Andrei Voronkov and Christoph Weidenbach say they do not.
      I'm looking for help on this.
\end{packed_itemize}

Given ways to interpret a formula wrt a interpretation, models can be verified.
The TPTP World uses the theorem proving approach to verification: a verification obligation is
formed with the interpretation-formulae or Herbrand-formulae as axioms, and the problem formulae
as a conjecture (when verifying a countermodel, the conjecture of the problem is negated).
The verification obligation is then discharged using a trusted theorem prover.
This processes is implemented in the AGMV model verifier \cite{SS+23-LPAR}, available in the 
SystemOnTSTP \cite{Sut07-CSR} web interface 
\href{https://www.tptp.org/cgi-bin/SystemOnTSTP}{{\tt www.tptp.org/cgi-bin/SystemOnTSTP}}.

%--------------------------------------------------------------------------------------------------
\section{The New Format for Tarskian Interpretations}
\label{NewTarskian}

The new TPTP format for interpretations uses interpretation-formulae.
Examples of single Tarskian interpretation-formulae are in Appendices~\ref{FOF_Finite.s}, 
\ref{TFF_Finite.s}, \ref{TFF_Integer.s}, \ref{TFF_Peano.s} and~\ref{THF_Finite.s}, illustrating 
the components described below. 
Tarskian interpretations split over multiple interpretation-formulae, as in 
Appendices~\ref{FOF_Finite_Medium.s}, \ref{FOF_Finite_Fine.s}, \ref{TFF_Finite_Medium.s}, 
\ref{TFF_Finite_Fine.s}, and~\ref{THF_Finite_Medium.s}, are explained in 
Section~\ref{NewTarskianSplit}.
Tarskian interpretations that are compacted using quantification, as in \ref{TFF_Finite_Compact.s}
and~\ref{THF_Finite_Compact.s}, are explained in Section~\ref{NewTarskianCompact}.
Examples of interpretation-formulae that induce Herbrand interpretations, as in 
Appendices~\ref{FOF_Saturation.s}, and~\ref{FOF_Formulae.s}, are explained in 
Section~\ref{NewHerbrand}.

\vspace*{1em}
\noindent
A single Tarskian interpretation-formula is a conjunction of components:
\begin{packed_itemize}
\item A domain for each of the types in the problem formulae.
\item Interpretation of the function symbols, as equalities whose left-hand sides are formed from 
      symbols applied to domain elements, and whose right-hand sides are domain elements.
\item Interpretation of the predicate symbols, as literals formed from symbols applied
      to domain elements; positive literals are {\em true} and negative literals are {\em false}.
\end{packed_itemize}

Note that function and predicate symbols are interpreted by directly applying them to domain
elements (ala \cite[\S5.3.4]{Gal15}), rather than the traditional presentation of interpretations
that introduces a new interpretation function for each function/predicate symbol 
\cite[\S5.3.2]{Gal15}.
This approach simplifies the representation of interpretations, and makes interpretation of 
formulae using theorem proving more tractable.
% The explanations and examples that follow will convince you of that!

%--------------------------------------------------------------------------------------------------
\subsection{FOF Tarskian Interpretations}
\label{NewTarskianFOF}

A FOF interpretation has a single untyped (or, if you will, of type {\tt \$i}) domain whose
elements are new (not appearing in the problem) ground terms. 
The distinctness of the domain elements must be made explicit, either with pairwise inequalities, 
or with the {\tt \$distinct} predicate, or by using {\tt "distinct object"}s as domain elements 
(they are known to be distinct (and often not used in problems, which makes them easy 
to identify as domain elements)).
Appendix~\ref{FOF_Finite.p} shows a problem that has a countermodel with a finite domain, 
shown in Appendix~\ref{FOF_Finite.s}.
Appendix~\ref{FOF_Finite.s} uses {\tt "distinct object"}s for domain elements.
Appendix~\ref{FOF_Finite.s.p} shows the verification problem for the interpretation.

%--------------------------------------------------------------------------------------------------
\subsection{TFF and THF Tarskian Interpretations}
\label{NewTarskianTFFTHF}

For TFF and TXF interpretation-formulae, each type of the problem has a domain, which can be of 
a new type.
If the problem type is different from the domain type then function and predicate symbols from 
the problem cannot be directly applied to domain elements of the type that corresponds to the type
of the argument position.
To resolve this {\em type-promotion} bijection are used to ``convert'' domain elements to terms 
of the corresponding type, thus keeping the interpretation formula well-typed -- See
Appendix~\ref{TFF_Finite.s} for examples.
The interpretation formula is preceded by the necessary type declarations:
\begin{packed_itemize}
\item The types in the problem (except defined types, e.g., {\smalltt{\$int}}).
\item The types of the domains (except defined types).
\item The types of type-promotion functions.
\item The types of the domain elements.
\end{packed_itemize}
Then, (following along in Appendices~\ref{TFF_Finite.s}, \ref{TFF_Integer.s}, and~\ref{TFF_Peano.s} 
might help here) each domain is a conjunction of:
\begin{packed_itemize}
\item The domain for each problem type, as a formula that makes the type-promotion function a 
      surjection (unless it is unnecessary because the domain type is defined and is the same as 
      the problem type, e.g., both are {\smalltt{\$int}}), e.g., in 
      Appendix~\ref{TFF_Finite.s}~\ldots\\
      \hspace*{0.5cm}\smalltt{! [C: cat] : ? [DC: d\_cat] : C = d2cat(DC)}
\item The elements of each domain (unless implicit from their defined type, as in
      Appendix~\ref{TFF_Integer.s}). 
      If the domain is finite this is a universally quantified disjunction of equalities whose 
      right-hand sides are the terms, e.g., in Appendix~\ref{TFF_Finite.s}~\ldots\\
      \hspace*{0.5cm}\smalltt{! [DC: d\_cat]: ( DC = d\_garfield | DC = d\_arlene | DC = d\_nermal )}\\
      If the domain is infinite this is an existentially quantified formula that captures an 
      infinite disjunction of equalities, e.g., in Appendix~\ref{TFF_Peano.s}~\ldots\\
      \hspace*{0.5cm}\smalltt{! [I: peano] : ( I = zero | ? [P: peano] : I = s(P) )}
\item Specification of the distinctness of the domain elements (unless implicit from their
      defined type);
\item A formula making the type-promotion function an injection,
      % (unless the type of the domain is the same as the type in the formula), 
      which together with the surjectivity makes it a bijection.
\end{packed_itemize}
The interpretation of symbols applies the corresponding type-promotion function to each domain
element argument so that that signatures of the symbols are respected.

\begin{packed_itemize}
\item Appendix~\ref{TFF_Finite.s} shows a TF0 interpretation-formula with finite domains -- it is a 
      countermodel for the problem in Appendix~\ref{TFF_Finite.p}.
      The comments show which parts of the formula specify what aspects of the interpretation.
      Appendix~\ref{TFF_Finite.s.p} shows the verification problem for the interpretation.
\item Appendix~\ref{TFF_Integer.s} shows a TF0 interpretation-formula with an integer domain -- it 
      is a model for the problem in Appendix~\ref{TFF_Infinite.p}.
      Note that $\bullet$~the defined type {\smalltt{\$int}} is the domain type for the formula 
      type {\smalltt{person}}, so that there is no specification of the domain elements and their 
      distinctness; $\bullet$~universal quantification is used for the interpretation of function 
      and predicate symbols for an infinite number of argument tuples; $\bullet$~the 
      interpretation of function and predicate symbols is not given for argument tuples with 
      negative integers, i.e., this is an example of a partial interpretation.
      Appendix~\ref{TFF_Integer.s.p} shows the verification problem for the interpretation.
\item Appendix~\ref{TFF_Peano.s} shows a TF0 interpretation-formula with an infinite term domain 
      -- it is a model for the problem in Appendix~\ref{TFF_Infinite.p}.
      Appendix~\ref{TFF_Peano.s.p} shows the verification problem for the interpretation.
\end{packed_itemize}

%--------------------------------------------------------------------------------------------------
\subsection{Tarskian Interpretations Split over Multiple Formulae}
\label{NewTarskianSplit}

The new TPTP format for Tarskian interpretations described thus far puts all the information 
about the interpretation in a single interpretation formula.
In some situations it is useful to separate the various components, e.g., the domains could be
separated from the symbol mappings.
The new TPTP format for Tarskian interpretations offers separation in a flexible way, at medium
and fine grained levels, using annotated formula subroles.

At the medium grained level the {\tt interpretation} role can be extended withe the subroles
{\tt domain} and {\tt mapping}.
Two (or more) interpretation formulae are used, each containing the corresponding parts of the 
interpretation.
Appendices~\ref{FOF_Finite_Medium.s} and \ref{TFF_Finite_Medium.s} show examples.
Note that each role and role-subrole pair can be used multiple times according to need, e.g.,
in Appendix~\ref{FOF_Finite_Medium.s} there are two {\tt interpretation-mapping} annotated
formulae, one for functions and one for predicates.

At the fine grained level the interpretation formula can be split into the individual domains
and symbol mappings, with the {\tt domain} and {\tt mapping} subroles given arguments to indicate
which domain or symbol mapping is recorded.
For {\tt domain}s the arguments are the domain from the problem and the domain in the 
interpretation.
For {\tt mapping}s the arguments are the symbol and its domain result type.
Appendices~\ref{FOF_Finite_Fine.s} and \ref{TFF_Finite_Fine.s} show examples.

The medium and fine grained splitting can be mixed, as is done in 
Appendix~\ref{THF_Finite_Medium.s}.

%--------------------------------------------------------------------------------------------------
\subsection{Tarskian Interpretations Compacted using Quantification}
\label{NewTarskianCompact}

In the context of using theorem proving to evaluate formulae wrt an interpretation represented
by interpretation-formulae, it is possible to use the full logical language in the interpretation 
formula.
This enable some compaction of the interpretation-formulae.
Appendices~\ref{TFF_Finite_Compact.s} and~\ref{THF_Finite_Compact.s} show examples.
In Appendix~\ref{TFF_Finite_Compact.s} note how universal quantification is used to map
{\tt loves} to {\tt d\_garfield} for all {\tt d\_cats}~\ldots \\
\hspace*{0.5cm}\smalltt{! [DC: d\_cat] : ( loves(d2cat(DC)) = d2cat(d\_garfield) )} \\
and in Appendix~\ref{THF_Finite_Compact.s} universal quantification is used to say that for all
functions of type {\tt beverage > syrup} and all {\tt syrups} applying the function to
{\tt d\_coffee} and the syrup results in {\tt d\_coffee}~\ldots \\
\hspace*{0.5cm}\smalltt{! [F: beverage > syrup > beverage,S: syrup] :} \\
\hspace*{1.0cm}\smalltt{( ( F @ ( d2beverage @ d\_coffee ) @ ( d2syrup @ S ) )} \\
\hspace*{1.0cm}\smalltt{= ( d2beverage @ d\_coffee ) )}

%--------------------------------------------------------------------------------------------------
\subsection{Saturations and Herbrand-formulae}
\label{NewHerbrand}

Saturations and Herbrand-formulae induce sets of Herbrand interpretations.
The new format can be used for these forms, although the actual Herbrand interpretations are
not easy to extract.
To indicate that the interpretation-formulae are intended to induce Herbrand interpretations they
can be given the role {\tt interpretation-herbrand}.
Appendices~\ref{FOF_Saturation.s} and~\ref{FOF_Formulae.s} show examples.
Appendices~\ref{FOF_Saturation.s.p} and~\ref{FOF_Formulae.s.p} show the verification problems for 
the interpretations.

%--------------------------------------------------------------------------------------------------
\section{The Format for Kripke Interpretations}
\label{NewKripke}
 
The format for Kripke interpretations also uses interpretation-formulae.
Examples of single Kripke interpretation formulae are in 
Appendices~\ref{NTF_Finite-Finite-Global.s}~and~\ref{NTF_Finite-Finite-Local.s}, 
illustrating the components described next. 
As was noted in Section~\ref{Have} Kripke interpretations can have either a finite or an
infinite number of worlds, and the Tarskian interpretations in a world can be finite Tarskian,
infinite Tarskian, or Herbrand interpretations.
I don't really have a firm handle on the cases with an infinite number of worlds, but I have some
tentative examples that show they can be represented in the new format.
For now the format is exp,ained a with examples that have a finite number of worlds and finite
Tarskian Tarskian interpretations in the worlds.

Four new defined symbols are used in Kripke interpretation-formulae:
\begin{packed_itemize}
\item A new defined type {\tt \$world} is used for the worlds of the interpretation.
      Syntactically distinct constants of type {\tt \$world} are known to be unequal. 
\item A new defined constant {\tt \$local\_world} with type {\tt \$world} is used to specify the
      world in which a local (the default) conjecture is proved. 
\item A new defined predicate {\tt \$accessible\_world}, with type {\tt (\$world * \$world) > \$o},
      is used to specify the accessibility relation between worlds.
\item A new defined predicate {\tt \$in\_world}, with type {\tt (\$world * \$o) > \$o}, is used for 
      specifying the Tarskian interpretations in each of the worlds.
      The use of this predicate with a boolean arguments means the interpretation-formulae is
      written in TXF or THF.
\end{packed_itemize}

In addition to the type declarations for the Tarskian interpretations in the worlds, the worlds
have to be declared of type {\tt \$world}, e.g., {\tt w1: \$world}.
Then, (following along in Appendices~\ref{NTF_Finite-Finite-Global.s}, 
and~\ref{NTF_Finite-Finite-Local.s} might help here) a single interpretation-formula is a 
conjunction of components:
\begin{packed_itemize}
\item Specification of the worlds, e.g., in Appendix~\ref{NTF_Finite-Finite-Global.s}~\ldots \\
      \hspace*{0.5cm}{\tt ! [W: \$world] : ( W = w1 | W = w2 )}
\item Until ATP systems and tools build it in, the distinctness of the worlds, e.g., 
      in Appendix~\ref{NTF_Finite-Finite-Global.s}~\ldots \\
      \hspace*{0.5cm}{\tt \$distinct(w1,w2)}.
\item The local world if any, e.g., in Appendix~\ref{NTF_Finite-Finite-Global.s}~\ldots \\
      \hspace*{0.5cm}{\tt \$local\_world = w1}
\item The accessibility relation, e.g.,~\ldots \\
      \hspace*{0.5cm}{\tt \$accessible\_world(w1,w1)} \\
      \hspace*{0.5cm}{\tt \$accessible\_world(w2,w2)} \\
      \hspace*{0.5cm}{\tt \$accessible\_world(w1,w2)}
\item For each world, the Tarskian interpretation within a {\tt \$in\_world} predicate, 
      e.g., in Appendix~\ref{NTF_Finite-Finite-Global.s}~\ldots \\
      \hspace*{0.5cm}{\tt \$in\_world(w1,}{\em the Tarskian interpretation}{\tt )}
\end{packed_itemize}

%--------------------------------------------------------------------------------------------------
\subsection{Finite-Finite Kripke Interpretations}
\label{NewKripkeFiniteFinite}

Appendix~\ref{NTF_Finite-Finite-Global.s} shows a TXF interpretation-formula with finite worlds
each of which has a finite Tarskian model -- it is a countermodel for the problem in 
Appendix~\ref{NTF_Finite-Finite-Global.p}.
Appendix~\ref{NTF_Finite-Finite-Global.s.p} shows the verification problem for the interpretation.

Appendix~\ref{NTF_Finite-Finite-Local.s} shows a TXF interpretation-formula with finite worlds
each of which has a finite Tarskian model -- it is a countermodel for the problem in 
Appendix~\ref{NTF_Finite-Finite-Local.p}.
This example illustrates how a local axiom (with role {\tt axiom-local} -- {\tt rotten(banana)})
needs to be satisfied only in the local world ({\tt w1}).
Appendix~\ref{NTF_Finite-Finite-Local.s.p} shows the verification problem for the interpretation.

%--------------------------------------------------------------------------------------------------
\subsection{Kripke Interpretations Split and Compacted}
\label{NewKripkeSplitCompact}

In the same way that a Tarskian Tarskian interpretation-formula can be split to separate the
various components, a Kripke interpretation-formula can be split.
The {\tt interpretation} role can be extended with the subroles {\tt world} and {\tt in\_world}.
Appendix~\ref{NTF_Finite-Finite-Global_Medium.s} shows a medium gained split of 
Appendix~\ref{NTF_Finite-Finite-Global.s}.
An annotated formulae with the role {\tt interpretation-world} contains information about the
worlds of the interpretation.
Annotated formulae with the role {\tt interpretation-in\_world} provides information about the
Tarskian interpretation for the world named as the first argument of the {\tt in\_world} subrole.
The second argument of the {\tt in\_world} subrole tells what information is provided, using the
roles and subroles for Tarskian interpretation-formulae.
For example, in Appendix~\ref{NTF_Finite-Finite-Global_Medium.s} the annotated formula with
role-subrole {\tt interpretation-in\_world(w1,interpretation-domain)} defines the domain of the
Tarskian Tarskian interpretation in world {\tt w1}.
Splitting can be done in a flexible way, e.g., Appendix~\ref{NTF_Finite-Finite-Global_Fine.s}
shows a more fine grained split of Appendix~\ref{NTF_Finite-Finite-Global.s}.

In the same way that a Tarskian Tarskian interpretation-formula can be compacted, Kripke
interpretation-formulae can also be compacted, e.g., by factoring out common elements of
the Tarskian interpretations in the worlds.
Appendix~\ref{NTF_Finite-Finite-Global_Compact.s} shows a compacted version of
Appendix~\ref{NTF_Finite-Finite-Global.s}.
Note the universal quantification over the worlds, which avoids the duplication in the 
{\tt in\_world} terms.

%--------------------------------------------------------------------------------------------------
%% \section{Model Verification}
%% \label{Verification}
%% 
%% ATP systems are complex pieces of software, implementing complex calculi, with the end goal
%% being a sound implementation of a sound inference system whose output correctly corroborates the
%% result obtained.
%% The reality is that the complexity leads to incorrectness, and as such verification of ATP systems'
%% outputs is necessary. 
%% For theorem proving this means verifying the proof output \cite{Sut06}, and for model finding 
%% this means verifying the model output.
%% In the context of this work the model verification applies to the type declarations and 
%% the interpretation formula that represent the model found by the ATP system, and
%% has (at least) the following aspects:
%% \begin{packed_enumerate}
%% \item Are the type declarations and interpretation formula syntactically well-formed 
%%       and semantically well-typed?
%% \item Is the interpretation formula satisfiable?
%% \item Does the interpretation formula correctly represent the interpretation found by the 
%%       ATP system?
%% \item Is the interpretation represented by the interpretation formula a model for the given 
%%       formulae?
%% \end{packed_enumerate}
%% \noindent
%% These questions are answered as follows:
%% \begin{enumerate}
%% \item This can be confirmed using standard parsing and type checking tools, e.g., \cite{VS06,HR15}.
%% \item This can be empirically confirmed using a trusted model finder (in the same way the GDV 
%%       derivation verifier \cite{Sut06} uses the Otter system \cite{McC03-Otter} as a trusted 
%%       theorem prover).
%%       Confirming that the interpretation formula is satisfiable is almost certainly much 
%%       easier than finding the model itself, so the system used to check the satisfiability can 
%%       be weaker and more trusted than the system that found the model.
%% \item This cannot be confirmed, as that representation is internal to the ATP system that found
%%       the model.
%% \item In this work a ``semantic'' approach is taken, in which the problem formulae $\Phi$ are 
%%       proved from the interpretation formulae $\varphi$ using a trusted theorem prover. 
%%       $\varphi$ are given as axioms, and $\Phi$ as the conjecture to be proved.
%%       The soundness of this approach is provided by showing that if $\Phi$ can be proved from 
%%       $\varphi$ then the interpretation $I$ represented by $\varphi$ is a model of $\Phi$.
%%       This is shown for finite FOF interpretations in \cite{SS+23-LPAR}, and that theorem lifts 
%%       (we believe) naturally to TFF, and THF, but is technically complicated due to the 
%%       introduction of types and type promotion functions. 
%%       The extension to infinite domains is quite simple after that.
%% \end{enumerate}
      
%--------------------------------------------------------------------------------------------------
\section{Conclusion}
\label{Conclusion}

This paper 

This new TPTP format for interpretations provides very many options and features, and it is
expected that only the basic features will be adopted initially. 
For ATP systems that already output the old TPTP format for finite Tarskian interpretations
all that is needed is to replace the {\tt fi\_domain}, {\tt fi\_functors}, and {\tt fi\_predicates}
roles by {\tt interpretation}. 
To be more informative, replace {\tt fi\_domain} by {\tt interpretation-domain}, and replace
{\tt fi\_functors} and {\tt fi\_predicates} by {\tt interpretation-mapping}.

Section~\ref{Need} listed three needs for interpretations: 
\begin{packed_itemize}
\item Verifiability - the ability to verify a model by evaluating all the formulae it claims
      to model as $true$.
\item Tractability - the ability to evaluate a formula using some limited/reasonable amount of
      resources.
\item Comprehensibility - the extent to which the interpretation can be understood, typically
      by a human but possibly by a machine.
\end{packed_itemize}

In the light of the above (and hey, maybe there are more techniques than just those) I claim~\ldots
\begin{packed_itemize}
\item Finite interpretations represented by interpretation-formulae are typically verifiable, 
      tractable, and comprehensible.
\item Infinite interpretations represented by interpretation-formulae are verifiable, often 
      intractable, and might not be comprehensible.
\item Saturations (in interpretation-formulae) are verifiable, often intractable, and almost 
      always incomprehensible.
\item Herbrand-formulae (in interpretation-formulae) are verifiable, can be tractable, and can 
      be comprehensible.
\item The verifiability, tractability, and comprehensibility of Kripke interpretations varies
      between individual cases.
\item Use of the same language as the problem formulae for writing interpretation-formulae
      contributes to comprehensibility (but does not ensure it).
\end{packed_itemize}

This work is being extended 

%--------------------------------------------------------------------------------------------------
\bibliographystyle{plain}
\bibliography{Bibliography.bib}
%--------------------------------------------------------------------------------------------------
\newpage
\appendix

All these listings are available from
\href{https://github.com/GeoffsPapers/InterpretationFormat}{github.com/GeoffsPapers/InterpretationFormat}.

\section{FOF}
\label{FOF}

\subsection{FOF Problem with a Finite Countermodel}
\label{FOF_Finite.p}
\begin{small}
\verbatiminput{FOF_Finite.p}
\end{small}

\newpage
\subsection{FOF Finite Model for \ref{FOF_Finite.p}, Coarse Grained}
\label{FOF_Finite.s}
\begin{small}
\verbatiminput{FOF_Finite.s}
\end{small}

\newpage
\subsection{FOF Verification Problem for \ref{FOF_Finite.p} and \ref{FOF_Finite.s}}
\label{FOF_Finite.s.p}
\begin{small}
\verbatiminput{FOF_Finite.s.p}
\end{small}

\newpage
\subsection{FOF Finite Model for \ref{FOF_Finite.p}, Medium Grained}
\label{FOF_Finite_Medium.s}
\begin{small}
\verbatiminput{FOF_Finite_Medium.s}
\end{small}

\newpage
\subsection{FOF Finite Model for \ref{FOF_Finite.p}, Fine Grained}
\label{FOF_Finite_Fine.s}
\begin{small}
\verbatiminput{FOF_Finite_Fine.s}
\end{small}

\newpage
\subsection{FOF Saturation for \ref{FOF_Finite.p}}
\label{FOF_Saturation.s}
\begin{small}
\verbatiminput{FOF_Saturation.s}
\end{small}

\newpage
\subsection{FOF Formula Model for \ref{FOF_Finite.p}}
\label{FOF_Formulae.s}
\begin{small}
\verbatiminput{FOF_Formulae.s}
\end{small}

\newpage
\subsection{FOF Verification Problem for \ref{FOF_Finite.p} and \ref{FOF_Saturation.s}}
\label{FOF_Saturation.s.p}
\begin{small}
\verbatiminput{FOF_Saturation.s.p}
\end{small}

\newpage
\subsection{FOF Verification Problem for \ref{FOF_Finite.p} and \ref{FOF_Formulae.s}}
\label{FOF_Formulae.s.p}
\begin{small}
\verbatiminput{FOF_Formulae.s.p}
\end{small}

%--------------------------------------------------------------------------------------------------
\newpage
\section{TF0, Finite Interpretations}
\label{TF0Finite}

\subsection{TF0 Problem with a Finite Countermodel}
\label{TFF_Finite.p}
\begin{small}
\verbatiminput{TFF_Finite.p}
\end{small}

\newpage
\subsection{TF0 Finite Model for \ref{TFF_Finite.p}, Coarse Grained}
\label{TFF_Finite.s}
\begin{small}
\verbatiminput{TFF_Finite.s}
\end{small}

\newpage
\subsection{TF0 Verification Problem for \ref{TFF_Finite.p}}
\label{TFF_Finite.s.p}
\begin{small}
\verbatiminput{TFF_Finite.s.p}
\end{small}

\newpage
\subsection{TF0 Finite Model for \ref{TFF_Finite.p}, Medium Grained}
\label{TFF_Finite_Medium.s}
\begin{small}
\verbatiminput{TFF_Finite_Medium.s}
\end{small}

\newpage
\subsection{TF0 Finite Model for \ref{TFF_Finite.p}, Fine Grained}
\label{TFF_Finite_Fine.s}
\begin{small}
\verbatiminput{TFF_Finite_Fine.s}
\end{small}

\newpage
\subsection{TF0 Finite Model for \ref{TFF_Finite.p}, Compacted}
\label{TFF_Finite_Compact.s}
\begin{small}
\verbatiminput{TFF_Finite_Compact.s}
\end{small}

%--------------------------------------------------------------------------------------------------
\newpage
\section{TF0, Infinite Interpretations}
\label{TF0Infinite}

\subsection{TF0 Axioms with an Infinite Model}
\label{TFF_Infinite.p}
\begin{small}
\verbatiminput{TFF_Infinite.p}
\end{small}

\newpage
\subsection{TF0 Infinite Model for \ref{TFF_Infinite.p}, Integer Domain}
\label{TFF_Integer.s}
\begin{small}
\verbatiminput{TFF_Integer.s}
\end{small}

\newpage
\subsection{TF0 Infinite Model for \ref{TFF_Infinite.p}, Term Domain}
\label{TFF_Peano.s}
\begin{small}
\verbatiminput{TFF_Peano.s}
\end{small}

\newpage
\subsection{TF0 Verification Problem for \ref{TFF_Infinite.p} and \ref{TFF_Integer.s}}
\label{TFF_Integer.s.p}
\begin{small}
\verbatiminput{TFF_Integer.s.p}
\end{small}

\newpage
\subsection{TF0 Verification Problem for \ref{TFF_Infinite.p} and \ref{TFF_Peano.s}}
\label{TFF_Peano.s.p}
\begin{small}
\verbatiminput{TFF_Peano.s.p}
\end{small}

%--------------------------------------------------------------------------------------------------
\newpage
\section{TH0, Finite Interpretations}
\label{TH0Finite}

\subsection{TH0 Problem with a Finite Countermodel}
\label{THF_Finite.p}
\begin{small}
\verbatiminput{THF_Finite.p}
\end{small}

\newpage
\subsection{TH0 Finite Model for \ref{THF_Finite.p}, Coarse Grained}
\label{THF_Finite.s}
\begin{small}
\verbatiminput{THF_Finite.s}
\end{small}

\newpage
\subsection{TH0 Verification Problem for \ref{THF_Finite.p}}
\label{THF_Finite.s.p}
\begin{small}
\verbatiminput{THF_Finite.s.p}
\end{small}

\newpage
\subsection{TH0 Finite Model for \ref{THF_Finite.p}, Mixed Grained}
\label{THF_Finite_Medium.s}
\begin{small}
\verbatiminput{THF_Finite_Medium.s}
\end{small}

\newpage
\subsection{TH0 Finite Model for \ref{THF_Finite.p}, Compacted}
\label{THF_Finite_Compact.s}
\begin{small}
\verbatiminput{THF_Finite_Compact.s}
\end{small}

%--------------------------------------------------------------------------------------------------
\newpage
\section{NXF, Finite Worlds with Finite Interpretations}
\label{NXF}

\subsection{NTF Problem with Global Axioms, \\ with a Finite-Finite Countermodel}
\label{NTF_Finite-Finite-Global.p}
\begin{small}
\verbatiminput{NTF_Finite-Finite-Global.p}
\end{small}

\newpage
\subsection{TXF Finite-Finite Model for \ref{NTF_Finite-Finite-Global.p}}
\label{NTF_Finite-Finite-Global.s}
\begin{small}
\verbatiminput{NTF_Finite-Finite-Global.s}
\end{small}

\newpage
\subsection{Verification Problem for \ref{NTF_Finite-Finite-Global.p} and 
\ref{NTF_Finite-Finite-Global.s}}
\label{NTF_Finite-Finite-Global.s.p}
\begin{small}
\verbatiminput{NTF_Finite-Finite-Global.s.p}
\end{small}

\newpage
\subsection{NTF Problem with Global and Local Axioms, \\
            with a Finite-Finite Countermodel}
\label{NTF_Finite-Finite-Local.p}
\begin{small}
\verbatiminput{NTF_Finite-Finite-Local.p}
\end{small}

\newpage
\subsection{TXF Finite-Finite Model for \ref{NTF_Finite-Finite-Local.p}}
\label{NTF_Finite-Finite-Local.s}
\begin{small}
\verbatiminput{NTF_Finite-Finite-Local.s}
\end{small}

\newpage
\subsection{Verification Problem for \ref{NTF_Finite-Finite-Local.p} and
\ref{NTF_Finite-Finite-Local.s}}
\label{NTF_Finite-Finite-Local.s.p}
\begin{small}
\verbatiminput{NTF_Finite-Finite-Local.s.p}
\end{small}

\newpage
\subsection{TXF Finite-Finite Model for \ref{NTF_Finite-Finite-Global.p}, Medium Grained}
\label{NTF_Finite-Finite-Global_Medium.s}
\begin{small}
\verbatiminput{NTF_Finite-Finite-Global_Medium.s}
\end{small}

\newpage
\subsection{TXF Finite-Finite Model for \ref{NTF_Finite-Finite-Global.p}, Fine Grained}
\label{NTF_Finite-Finite-Global_Fine.s}
\begin{small}
\verbatiminput{NTF_Finite-Finite-Global_Fine.s}
\end{small}

\newpage
\subsection{TXF Finite-Finite Model for \ref{NTF_Finite-Finite-Global.p}, Compacted}
\label{NTF_Finite-Finite-Global_Compact.s}
\begin{small}
\verbatiminput{NTF_Finite-Finite-Global_Compact.s}
\end{small}

%--------------------------------------------------------------------------------------------------
\end{document}
%--------------------------------------------------------------------------------------------------
